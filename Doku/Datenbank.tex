
In diesem Abschnitt werden die Schritte zur Installation von PostgreSQL und die Einrichtung eines Datenbankservers auf Ihrem Computer beschrieben. Außerdem werden die Tabellenstruktur und die Schema-Definitionen für die Datenbank vorgestellt.

\subsection{Installation von PostgreSQL}

Um PostgreSQL auf einem Windows-System zu installieren, müssen die folgenden Schritte ausgeführt werden:

\begin{enumerate}
\item Unter der offiziellen PostgreSQL-Website \url{https://www.postgresql.org/download/windows/} müssen den Anweisungen für Windows befolgt werden.
\item Installationsprogramm starten und den Anweisungen auf dem Bildschirm folgen, um die Installation abzuschließen.
\item Die Option pgAdmin4 auswählen, um pdAdmin4 für die Verwaltung der Datenbank zu installieren.
\end{enumerate}

\subsection{Einrichtung eines Datenbankservers}

Nachdem PostgreSQL erfolgreich installiert wurde, kann ein Datenbankserver auf eingerichtet werden. Hierfür wurde pgAdmin4 benutzt:

\begin{enumerate}
\item In der linken Navigationsleiste wird auf \enquote{Servers/PostgreSQL/Databases} geklickt.
\item Durch einen Rechtsklick auf \enquote{Databases} wird eine neue Datenbank erstellt oder im Terminal wird der Befehl \enquote{CREATE DATABASE <datenbankname>;} ausgeführt. Standardmäßig wird eine Datenbank namens "postgres" erstellt.
\item Standardmäßig läuft der Server auf Localhost auf Port 5432.
\end{enumerate}

\subsection{Tabellenstruktur und Schema-Definitionen}

In unserer Beispielanwendung werden zwei Tabellen verwendet: die Tabelle "users" und die Tabelle "working\_days". Die Struktur und Definitionen dieser Tabellen sind wie folgt:

\begin{minted}{sql}
-- Tabelle "users"
CREATE TABLE users (
id serial primary key,
first_name varchar not null,
last_name varchar not null
);

-- Tabelle "working_days"
CREATE TABLE working_days (
id SERIAL PRIMARY KEY,
starting_time varchar ,
ending_time varchar ,
working_hours varchar ,
user_id INTEGER REFERENCES "users"(id) ON DELETE CASCADE
);
\end{minted}

Die Tabelle \enquote{users} enthält drei Spalten: \enquote{id}, \enquote{first\_name} und \enquote{last\_name}. Die \enquote{id}-Spalte ist der Primärschlüssel der Tabelle. 
Die Tabelle \enquote{working\_days} enthält fünf Spalten: \enquote{id}, \enquote{starting\_time}, \enquote{ending\_time}, \enquote{working\_hours} und \enquote{user\_id}. Die \enquote{id}-Spalte ist auch hier der Primärschlüssel. Die Spalte \enquote{user\_id} ist ein Fremdschlüssel, der auf die \enquote{id}-Spalte der Tabelle \enquote{users} verweist und das Löschen von Datensätzen in der Tabelle \enquote{users} automatisch das Löschen der entsprechenden Datensätze in der Tabelle \enquote{working\_days} auslöst.
Die Datentypen von \enquote{starting\_time}, \enquote{ending\_time} und \enquote{working\_hours} sind auf VARCHAR gesetzt. Das Handling der Arbeitszeitrechnung wird im Backend in Rust durchgeführt. 

Mit diesen Schritten und Definitionen sollte die PostgreSQL-Datenbank einsatzbereit sein.