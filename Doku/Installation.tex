
    \subsection{Rust-Installation}
    Um Rust auf einem Windows-System zu installieren, müssen die folgenden Schritte ausgeführt werden:
    \begin{enumerate}
        \item Unter der offiziellen Rust-Website \url{https://www.rust-lang.org/learn/get-started} müssen den Anweisungen für Windows befolgt werden.
        \item Mit den Standardoptionen kann man das Packet installieren.
        \item Um sicherzustellen, dass Rust erfolgreich installiert wurde, kann man im Terminal den Befehl \texttt{rustc --version} ausführen. Die installierte Rust-Version soll angezeigt werden.
    \end{enumerate}

    \subsection{Installation von C++ Build Tools}
    Um Rust auf Windows zu installieren, muss man zunächst C++ Build Tools installieren:
    \begin{enumerate}
        \item Zur offiziellen Microsoft-Website für Visual C++ Build Tools unter \url{https://visualstudio.microsoft.com/de/visual-cpp-build-tools/} gehen.
        \item Auf den Download-Button klicken, um das Installationsprogramm herunterzuladen.
        \item Das Installationsprogramm starten und die Option "Desktop development with C++" auswählen.
        \item Anweisungen im Installationsprogramm befolgen und die Standardoptionen akzeptieren.
    \end{enumerate}
    
    \subsection{Einrichtung einer Entwicklungsumgebung}
    Nach der Installation von Rust und den erforderlichen Abhängigkeiten wird eine Entwicklungsumgebung eingerichtet:
    \begin{enumerate}
        \item Als Entwicklungsumgebung (IDE)wird in diesem Projekt mit Visual Studio Code gearbeitet.
        \item Das Plugin Rust Analyzer von The Rust Programming Language installieren.
        \item Ein neues Rust-Projekt mit dem Befehl \texttt{cargo new RustProjektWebanwendung} erstellen.
    \end{enumerate}
